\documentclass{mc2015}

%%%%%%%%%%%%%%%%%%%%%%%%%%%%%%%%%%%%%%%%%%%%%%%%%%%%%%%%%%%%%%%%%%%%%
\usepackage[T1]{fontenc}         % Use T1 encoding instead of OT1
\usepackage[utf8]{inputenc}      % Use UTF8 input encoding
\usepackage{microtype}           % Improve typography
\usepackage{booktabs}            % Publication quality tables
\usepackage{amsmath}
\usepackage{graphicx}
\usepackage{float}
\usepackage[load-configurations=binary,exponent-product=\cdot]{siunitx}
\usepackage[colorlinks,breaklinks]{hyperref}
\hypersetup{linkcolor=black, citecolor=black, urlcolor=black}

\def\equationautorefname{Eq.}
\def\figureautorefname{Fig.}

%%%%%%%%%%%%%%%%%%%%%%%%%%%%%%%%%%%%%%%%%%%%%%%%%%%%%%%%%%%%%%%%%%%%%
\authorHead{Romano, Siegel, and Rahaman}
\shortTitle{Influence of the Memory Subsystem on Monte Carlo Code Performance}

%%%%%%%%%%%%%%%%%%%%%%%%%%%%%%%%%%%%%%%%%%%%%%%%%%%%%%%%%%%%%%%%%%%%%
\begin{document}

\title{Influence of the Memory Subsystem on Monte Carlo Code Performance}

\author{Paul K. Romano}
\affil{Bechtel Marine Propulsion Corporation, Knolls Atomic Power Laboratory
  \\ P.O. Box 1072, Schenectady, NY 12301, United States
  \\ paul.romano@unnpp.gov}

\author{Andrew R. Siegel}
\author{Ronald O. Rahaman}
\affil{Argonne National Laboratory, Theory and Computing Sciences \\ 9700 S
  Cass Ave., Argonne, IL 60439, United States \\ siegela@mcs.anl.gov}

\maketitle

\begin{abstract}
In this study, a detailed look at how miss rates and latencies in a multi-level
memory hierarchy can have significant effects on the performance of a Monte
Carlo code is presented. Simulations of the Monte Carlo performance benchmark
were run, and hardware performance counters were collected using the
Performance API (PAPI). The results of the simulations and an accompanying
analysis suggest that for light-water reactor depletion problems, the most
important factor that determines performance is the effective memory latency
accounting for characteristics of the L2 cache, L3 cache, and main
memory. Observed performance in multi-socket NUMA architectures was also
explained by the performance counters collected.

\emph{Key Words}: Monte Carlo, memory, cache, NUMA, OpenMP
\end{abstract}

%%%%%%%%%%%%%%%%%%%%%%%%%%%%%%%%%%%%%%%%%%%%%%%%%%%%%%%%%%%%%%%%%%%%%
\section{Introduction}

Over the last decade, a paradigm shift has occurred in microprocessor design,
where increases in performance are now primarily obtained via greater
thread-level and data-level parallelism rather than by increases in CPU clock
frequencies as evidenced by the current generation of multi-core, pipelined,
multiple-issue processors. While the raw floating point performance of
processors has grown steadily, the performance of the memory subsystem has not
kept up to pace and as a result, many codes are now limited not by floating
point operations per second (FLOPs) but rather by memory loads and stores. One
key example is Monte Carlo (MC) neutral particle transport as used for nuclear
reactor calculations. These calculations are particularly challenging for a
number of reasons: 1) the memory requirements have the potential to exceed the
memory available on a single node, 2) the complex memory access patterns
typically result in a high cache miss rate, and 3) the level of spatial fidelity
needed requires extremely long time-to-solution for most problems of interest.

Many recent studies have focused on understanding and offering solutions to
these challenging aspects of MC reactor simulations. Algorithmic improvements
have helped to achieve linear scaling on distributed-memory
architectures~\cite{nse-romano-2012, ane-romano-2013} , and various studies have
shown that the total memory requirements can be decomposed effectively using
either domain decomposition~\cite{jcp-siegel-2012, jcp-siegel-2013,
  physor-horelik-2014, pc-horelik-2014} or data
decomposition~\cite{jcp-romano-2013}. With most of the barriers to effective
utilization of distribution-memory architectures having been overcome, the focus
is now shifting to scaling and performance issues for on-node, shared-memory
architectures. Making effective use of multi- and many-core processors will be
considerably more difficult due to the greater complexity of memory hierarchies
and the fact that they are generally not programmable, i.e. the programmer does
not have direct control over the locality of data within the memory hierarchy
as is the case in distributed-memory systems. Indeed, initial studies
demonstrate that sub-optimal scaling is observed for MC methods on a variety
of hardware~\cite{ijhpca-siegel-2014}.

Researchers are beginning to make inroads in understanding the reasons for the
degradation of scaling on shared-memory systems. Of particular interest are the
studies of Tramm et al. demonstrating that the key performance barrier for MC
reactor calculations on multi-core processors is contention for shared-memory
resources. Through the development of the XSBench mini-app, Tramm and
Siegel show that many hardware counters relating to the memory subsystem are
directly correlated to performance degradation for multi-threaded MC
simulations~\cite{physor-tramm-2014, ane-tramm-2014}. In the present work, we
seek to address a number of shortcoming in these studies, namely:
\begin{itemize}
\item The previous studies did not account for tallies that would be required
  for a reactor depletion calculation; various reaction rates need to be
  tallied for each nuclide in each unique depletable region in a reactor---for a
full core analysis, each depletable region would consist of a segment of a fuel pin.
\item While the studies demonstrated correlation between the scaling
  degradation and various hardware counters, no explanation was offered for
  the exact mechanism of the degradation.
\item Performance counter measurements were performed on the XSBench mini-app,
  but similar measurements were not shown for the full application it is meant
  to mimic, OpenMC.
\end{itemize}
As will be shown. the results also help to explain performance trade-offs for
using fewer or greater numbers of processes and threads within a single node.

\subsection{MC Particle Transport Simulations}

MC neutral particle transport methods are typically formulated by using the
classical history method, where individual particles are followed one-by-one
from birth to death. This approach lends itself naturally to a parallel
implementation where each processing element tracks a different particle. This
can be done at either the process- or thread-level (or both). In the former
case, the simplest approach is to fully replicate the data structures in the
physical address space of each process, provided enough on-node memory is
available. In the latter case, most of the data is shared between different
threads with only a few key variables replicated for each thread. The amount of
memory required by each process largely depends on the characteristics of the
problem being run, with the following generic categories of data:
\begin{itemize}
\item \emph{Geometry and materials} --- An abstract model of the shape and
  composition of each unique region within the geometry must be held in
  memory. In most MC codes, the geometry is represented using combinatorial
  geometry, where regions of space are defined as the union, difference, and
  intersection of curved surfaces. Each region of space must also be assigned a
  material, defined by its physical characteristics which, at the very least,
  includes the nuclides and their corresponding densities.
\item \emph{Cross sections} --- Each unique nuclide in the problem must be
  accompanied by a set of interaction data that describe the likelihood of
  various reactions and their outcomes. This data is used to sample the
  distance to next collision as well as to select a nuclide and reaction within
  a material for sampling collision physics.
\item \emph{Tallies} --- This refers to a collection of pre- and user-defined
  physical quantities that are accumulated during the simulation. As an
  example, the user may wish to know the fission reaction rate within a
  particular region of the problem.
\end{itemize}
Unfortunately, due to the stochastic nature of particle transport, there is no
consistent pattern for how the various parts of the data in memory are
accessed. To understand why this is detrimental to performance, a basic
understanding of the memory hierarchy of modern computer architectures is
necessary.

\subsection{Memory and Cache Hierarchy}

When a process is executed, the program data at any point of time during
execution is stored in main memory. Physically, main memory is implemented in
DRAM (dynamic random access memory) chips, which at the time of writing
typically hold on the order of gigabytes of data. However, the high latency of
accessing data, in the hundreds of clock cycles, means that it is desirable to
avoid having to constantly transfer data to and from main memory. Thus, most
CPUs implement a series a caches: smaller but faster memories that hold
recently-used data. Most modern multi-core CPUs have two or three levels of
caches; the fastest (and smallest) is referred to as the L1 (level 1) cache and
subsequent caches are ordered in terms of increasing latency and size. When the
CPU issues a load instruction, the hardware first determines whether the block
of memory requested is already in the highest level (L1) cache. If it is, this
is referred to as a L1 \emph{cache hit}. Similarly, if the data is not present,
it's called a L1 \emph{cache miss}. In the event of a cache miss, the next
level of cache is checked for the requested data, and the process is repeated
until the data is found. The fraction of accesses that result in a cache hit or
miss is called the \emph{hit rate} and \emph{miss rate}, respectively.

Ideally, one would like all the memory to always be available in the highest
level, lowest latency cache. However, the L1 cache size is typically less than
a megabyte, whereas the memory needed to run a MC simulation can be tens or
hundreds of gigabytes. Given that the data is accessed more-or-less randomly,
the cache miss ratios can be very high, meaning that many memory accesses would
incur the latency penalty of accessing the L3 cache or, even worse, main
memory. As we will show in the results, the observed performance of a MC
transport simulation is very closely related to the effective memory latency
accounting for all levels of the memory hierarchy. This has important
implications for MC run strategies.

The performance of the cache hierarchy depends not only on the sizes and speeds
of the cache, but also on how data is managed and mapped between different
levels. This is characterized by the \emph{associativity}, \emph{replacement
policy}, and whether write misses are handled via write-through or
write-back. In modern processors, a specialized cache called the
translation lookaside buffer (TLB) that handles virtual-to-physical address
translation is also used. A thorough discussion of these implementation details
is beyond the scope of the current work.

\section{Methods}

The goal of the present work is to characterize the impact of a multi-level
memory hierarchy on the performance of a Monte Carlo neutron transport
simulation. To do this, we will look at the performance of the OpenMC Monte
Carlo particle transport code~\cite{ane-romano-2013, ane-romano-2014}, an
open-source continuous-energy transport code that uses combinatorial
geometry. OpenMC, while it does not have as rich a set of capabilities as more
mature codes like MCNP~\cite{lanl-goorley-2014} or
MC21~\cite{ane-griesheimer-2014}, uses the same methods and interaction data for
tracking particles; thus, conclusions about the behavior of OpenMC should apply
equally well to other production codes. OpenMC implements distributed-memory
parallelism via MPI and shared-memory parallelism via OpenMP and has
demonstrated excellent distributed-memory scaling on leadership class
supercomputers~\cite{ane-romano-2013}.

Following in the same vein as previous works, we will restrict our tests to a
single reactor benchmark---the Monte Carlo performance
benchmark~\cite{mc-hoogenboom-2011}, more commonly referred to as the
Hoogenboom-Martin benchmark after the names of its original authors. We will
herein refer to it as the H-M benchmark. This benchmark captures most of the
important algorithmic and computational complexities of a pressurized water
reactor problem while not being overly cumbersome to model.

To relate the observed performance in OpenMC to measurements of accesses and
miss rates at various levels of the memory hierarchy, a number of techniques
could be employed. One approach is to use a cache simulation tool like
Cachegrind~\cite{cachegrind}, part of the Valgrind utility, that relies on
trace-driven simulation of the machine's cache. The benefits of this approach
are its ease of use, ability to simulate caches of different sizes, and the fact
that it can attribute cache misses and hits to lines of source code. However, it
is not without a number of shortcomings; to name a few, TLB cache hits and
misses due to speculative execution are not considered, all threads are
serialized such that the results may not be representative of the true
performance, and the simulated cache may not necessarily match the machine's
cache with respect to associativity and replacement policy. Instead, we have
chosen to use the Performance Application Programming Interface (PAPI), which
provides a set of library routines for directly instrumenting hardware
performance counters. Hardware counters are special-purpose registers that can
store the count of many hardware events with very low overhead. By using
hardware counters directly, we obtain a much more accurate accounting of cache
hits and misses and can combine measurements from threads running simultaneously
as well.The downside is that the code has to make calls to PAPI directly, and
the types and meanings of performance counters can vary considerably across
different architectures.

PAPI requires that the user request a set of hardware events to be
considered. In our simulations of OpenMC, we collected the following counters:
level 2 instruction cache accesses and misses (\texttt{PAPI\_L2\_ICA} and
\texttt{PAPI\_L2\_ICM}), level 2 data accesses and misses
(\texttt{PAPI\_L2\_DCA} and \texttt{PAPI\_L2\_DCM}), and level 3 cache total
accesses and misses (\texttt{PAPI\_L3\_TCA} and \linebreak
\texttt{PAPI\_L3\_TCM}).

%%%%%%%%%%%%%%%%%%%%%%%%%%%%%%%%%%%%%%%%%%%%%%%%%%%%%%%%%%%%%%%%%%%%%
\section{Results and Analysis}

To understand the performance of the OpenMC code on the H-M benchmark and how
it relates to the memory and cache hierarchy, a series of simulations was
performed and the aforementioned performance counters were collected. Four
different cases of the H-M benchmark were considered. The first case is the
standard H-M benchmark with no modifications. The second case is the same as
the first, except that reaction rates for four reactions were collected for
each nuclide over a 289 x 289 mesh. This was done to mimic the tallies that
would be necessary for a depletion simulation. The third and fourth cases were
modified to include 423 nuclide is the fuel region, with the third having no
tallies and the fourth having the same tallies as in the second case. The
number of nuclides in the fuel was increased as a depletion simulation would
require a large nuclide inventory. Given that the number of cross section
lookups is directly proportional to the number of nuclides, having a larger
nuclide inventory has a significant effect of performance. Let us refer to
cases 1 and 2 as small H-M and to cases 3 and 4 as large H-M.

For each of the four cases, OpenMC was run on two different nodes. The first
node contains two 10-core Intel Xeon E5-2680 v2 (Ivy Bridge-EP
microarchitecture) processors, each of which has private L1 and L2 instruction
and data caches and a shared L3 cache. The second node contains two 12-core
Intel Xeon E5-2680 v3 (Haswell-EP microarchitecture) processors, which also have
private L1 and L2 instruction and data caches and a shared L3 cache. While each
simulation was run so as to fully utilize all cores within the node, we varied
the number of processors and threads to analyze its effect of performance as
well.

\autoref{tab:small-ivy} shows the results for the small H-M benchmark on the Ivy
Bridge processor with and without tallies. Miss rates were calculated simply as
the number of misses over the number of accesses. The bandwidth utilization was
calculated as~\cite{physor-tramm-2014}
\begin{equation}
  \label{eq:bandwidth}
  \text{Bandwidth} = \frac{\text{\texttt{PAPI\_L3\_TCM}}\cdot\text{Cache line
      size}}{\text{Wall-clock time}}.
\end{equation}
In theory, rather than using the wall-clock time, it would also be possible to
use the CPU clock frequency and a performance counter for the number of cycles
to calculate the bandwidth. However, the Intel Xeon processors used in this
study take advantage of a feature called Turbo Boost where the clock frequency
changes dynamically based on the workload.

We see that the best performance is achieved with two MPI processes and eight
OpenMC threads. This can be explained by the following arguments. When 20
processes are used, each process has its own physical address space containing
the fully replicated problem data. Thus, even if two processes load the same
piece of data, e.g. a cross section for a particular nuclide at a particular
energy, at the same time, there is no guarantee that either will results in a L3
cache hit because the data are in separate address spaces. As more threads and
fewer processors are used, there is a higher likelihood that data used by
multiple threads will already be in the shared L3 cache, and thus we see that
the L3 miss rate drops substantially as more threads are added. Note however
that when we reach 20 threads, a separate effect is at play. As there are two
physical CPUs, each with its own main memory bank and shared L3 cache, when 20
threads are used, half of the memory accesses will refer to physical addresses
in the remote memory bank which results in a much higher L3 miss
rate. Furthermore, the latency to access the remote L3 cache and main memory is
considerably higher than the latency for the local memories\footnote{In this
  particular architecture, memory accesses to the remote L3 cache and main
  memory must travel through the Quick Path interconnect (QPI) bus}. Together,
the increase in the L3 miss rate and the effective memory latencies contribute
to a stark drop in the observed performance.
\begin{table}[htb]
  \centering
  \caption{Cache accesses and miss rates, bandwidth utilization, and observed
    performance for the small H-M benchmark on the Intel Xeon E5-2680 v2 (Ivy
    Bridge) processor. All quantities followed by $\ddagger$ are on a per
    particle basis.}
  \label{tab:small-ivy}
  \footnotesize{
  \begin{tabular}{l*{8}{r}}
    \toprule
    & \multicolumn{2}{c}{L2 Instruction} & \multicolumn{2}{c}{L2 Data} &
    \multicolumn{2}{c}{L3} \\
    \cmidrule(r){2-3} \cmidrule(r){4-5} \cmidrule(r){6-7}
    \parbox{1.5cm}{Processes /\\Threads} & Accesses$^\ddagger$
    & \parbox[c]{1.2cm}{\centering Miss\\Rate (\%)} & Accesses$^\ddagger$
    & \parbox[c]{1.2cm}{\centering Miss\\Rate (\%)} & Accesses$^\ddagger$
    & \parbox[c]{1.2cm}{\centering Miss\\Rate (\%)}
    & \parbox[c]{1.3cm}{\centering Bandwidth$^\ddagger$\\(MiB/s)}
    & \parbox[c]{1.8cm}{\centering Tracking Rate\\(neutron/s)} \\
    \midrule
    \multicolumn{9}{c}{\textbf{No Tallies}} \\
    \midrule
    20 / 1 & 1469 & 38.7 & 28776 & 27.1 & 8367 & 19.7 & 8648 & 85990 \\
    10 / 2 & 1481 & 38.2 & 30268 & 30.8 & 9887 & 8.6 & 4636 & 88852 \\
    4 / 5 & 1476 & 38.6 & 30506 & 33.3 & 10728 & 3.3 & 1921 & 89510 \\
    2 / 10 & 1485 & 37.2 & 30654 & 35.7 & 11490 & 1 & 643 & 89986 \\
    1 / 20 & 1640 & 38.6 & 31381 & 36.8 & 12189 & 9.4 & 3326 & 47730 \\
    \midrule
    \multicolumn{9}{c}{\textbf{Tallies}} \\
    \midrule
    20 / 1 & 5433 & 25.3 & 52918 & 24.4 & 14282 & 23.2 & 9162 & 45387 \\
    10 / 2 & 5472 & 25.9 & 54686 & 26.2 & 15767 & 13.9 & 6298 & 47114 \\
    4 / 5 & 5474 & 26.8 & 55376 & 27.4 & 16642 & 8.6 & 4173 & 48014 \\
    2 / 10 & 5462 & 25.8 & 55522 & 29.2 & 17624 & 6.2 & 3207 & 48265 \\
    1 / 20 & 5640 & 25.9 & 55025 & 30.8 & 18384 & 13 & 4438 & 30463 \\
    \bottomrule
  \end{tabular}
  }
\end{table}

Interestingly, while the L3 miss rate drops from about 20\% to 1\% going from
20 processes/1 thread to 2 processes/10 threads for the no tallies case in
\autoref{tab:small-ivy}, the performance increases only by a very small
amount. This is due to the fact the the L2 miss rate increases slightly with
increasing numbers of threads thereby offsetting the lower L3 miss rate. To
substantiate this, we can develop a metric for the effective total memory
latency per particle:
\begin{equation}
  \bar{\alpha} = N_{L2}\cdot \left ( (1-M_{L2})\alpha_{L2} + M_{L2}\cdot((1 -
  M_{L3})\alpha_{L3} + M_{L3} \alpha_{main}) \right )
\end{equation}
where $N_X$ is the total accesses per particle at the $X$ level (where $X$ can
be L2, L3 or main memory), $M_X$ is the miss rate at the $X$ level, and
$\alpha_X$ is the latency for a hit at the $X$ level. To calculate the effective
latency, we use some gross estimates from~\cite{intel} for memory latencies: 10
cycles of a L2 hit, 50 cycles for a local L3 hit, 100 cycles for a remote L3
hit, 170 cycles for local DRAM access, and 280 cycles for remote DRAM access. In
the 20 thread case, we assume that L3 and main memory accesses are split between
the local and remote processor. \autoref{tab:latency} shows the calculated
effective latencies for the small H-M benchmark. The expected speedup was
calculated assuming that the performance scales as the inverse of the effective
latency; therefore, the relative speedup is simply the ratio of effective
latencies. Comparing the expected speedup to the actual observed speedup, we see
that they are in remarkable agreement. The effective latency argument
successfully predicts a smaller speedup for the no tallies case as well as the
large drop in performance observed for the 20 thread runs.
\begin{table}[htb]
  \centering
  \caption{Effective latency and expected/actual speedup for the small H-M
    benchmark on the Intel Xeon E5-2680 v2 (Ivy Bridge) processor.}
  \label{tab:latency}
  \footnotesize{
  \begin{tabular}{l*{3}{r}}
    \toprule
    \parbox[c]{1.5cm}{\centering Processes /\\Threads}
    & \parbox[c]{2cm}{\centering Latency$^\ddagger$\\($10^3$ cycles)}
    & \parbox[c]{3cm}{\centering Expected speedup\\(compared to 20 / 1)}
    & \parbox[c]{3cm}{\centering Actual speedup\\(compared to 20 / 1)} \\
    \midrule
    \multicolumn{4}{c}{\textbf{No Tallies}} \\
    \midrule
    20 / 1 & 784 & 1.00 & 1.00  \\
    10 / 2 & 772 & 1.02 & 1.03  \\
    4 / 5 &  751 & 1.04 & 1.04 \\
    2 / 10 & 757 & 1.04 & 1.05  \\
    1 / 20 & 1227 & 0.64 & 0.56 \\
    \midrule
    \multicolumn{4}{c}{\textbf{Tallies}} \\
    \midrule
    20 / 1 & 15089 & 1.00 & 1.00 \\
    10 / 2 & 14381 & 1.05 & 1.06 \\
    4 / 5 &  12696 & 1.19 & 1.18 \\
    2 / 10 & 11526 & 1.31 & 1.30 \\
    1 / 20 & 17438 & 0.87 & 0.80 \\
    \bottomrule
  \end{tabular}
  }
\end{table}

\autoref{tab:large-ivy} shows the results for the large H-M benchmark on the
Ivy Bridge processor with and without tallies. Similar themes are seen as in
the small H-M benchmark where the maximum performance is obtained with 2
processes / 10 threads. Due to the code spending a large fraction of its time
collecting tallies, the L2 data cache accesses are much higher than for small
H-M along with a very high miss rate, resulting in a high number of L3 accesses
as well. Also note that the main memory bandwidth utilization becomes as high
as 21 GiB/s. Previous measurements~\cite{ane-tramm-2014} using the STREAM
benchmark demonstrated that the maximum sustained bandwidth available to a user
application on a similar platform is about 25--30 GiB/s. This lends further
credence to the argument that in our cases, OpenMC was primarily limited by
latency rather than available bandwidth.
\begin{table}[htb]
  \centering
  \caption{Cache accesses and miss rates, bandwidth utilization, and observed
    performance for the large H-M benchmark on the Intel Xeon E5-2680 v2 (Ivy
    Bridge) processor. All quantities followed by $\ddagger$ are on a per
    particle basis.}
  \label{tab:large-ivy}
  \footnotesize{
  \begin{tabular}{l*{8}{r}}
    \toprule
    & \multicolumn{2}{c}{L2 Instruction} & \multicolumn{2}{c}{L2 Data} &
    \multicolumn{2}{c}{L3} \\
    \cmidrule(r){2-3} \cmidrule(r){4-5} \cmidrule(r){6-7}
    \parbox{1.5cm}{Processes /\\Threads} & Accesses$^\ddagger$
    & \parbox[c]{1.2cm}{\centering Miss\\Rate (\%)} & Accesses$^\ddagger$
    & \parbox[c]{1.2cm}{\centering Miss\\Rate (\%)} & Accesses$^\ddagger$
    & \parbox[c]{1.2cm}{\centering Miss\\Rate (\%)}
    & \parbox[c]{1.3cm}{\centering Bandwidth$^\ddagger$\\(MiB/s)}
    & \parbox[c]{1.8cm}{\centering Tracking Rate\\(neutron/s)} \\
    \midrule
    \multicolumn{9}{c}{\textbf{No Tallies}} \\
    \midrule
    20 / 1 & 2228 & 78.5 & 227703 & 85.4 & 196259 & 21.6 & 21333 & 8261 \\
    10 / 2 & 2180 & 78.5 & 235104 & 85.8 & 203425 & 16.4 & 17845 & 8783 \\
    4 / 5 & 2153 & 79 & 234709 & 85.9 & 203321 & 9.4 & 11443 & 9767 \\
    2 / 10 & 2140 & 79.2 & 235053 & 85.8 & 203369 & 4.6 & 6107 & 10745 \\
    1 / 20 & 2252 & 78.1 & 235837 & 86 & 204619 & 6.2 & 5149 & 6626 \\
    \midrule
    \multicolumn{9}{c}{\textbf{Tallies}} \\
    \midrule
    20 / 1 & 7090 & 63.8 & 493521 & 68.4 & 342185 & 20.8 & 13654 & 3140 \\
    10 / 2 & 6807 & 63.7 & 452187 & 70.3 & 322207 & 15.5 & 12562 & 4127 \\
    4 / 5 & 6671 & 63.5 & 423042 & 70 & 300459 & 10.6 & 9735 & 5029 \\
    2 / 10 & 6627 & 63.5 & 413624 & 69.6 & 291903 & 7 & 6664 & 5339 \\
    1 / 20 & 6757 & 63.3 & 414357 & 71.4 & 299988 & 8 & 5969 & 4074 \\
    \bottomrule
  \end{tabular}
  }
\end{table}

Another factor that can potentially affect performance is MPI communication,
especially when a large number of tallies are used. While the simulations run
here were constructed so as to minimize the impact of network communication, it
still had a noticeable effect on the cases with tallies using 20 MPI
processes. In the large H-M case with tallies, almost half of the time was spent
accumulating tally data from the different MPI ranks. That being said,
this effect can be reduced to an negligible level by increasing the number of
particles simulated per batch, or using other techniques to reduce the
associated communication~\cite{trans-romano-2012}.

\autoref{tab:small-haswell} and \autoref{tab:large-haswell} show the results for
the small and large H-M benchmarks, respectively, on the node with two 12-core
Haswell processors with and without tallies. In this case, the best performance
is obtained for 4 processes / 6 threads. This is a result of a unique snoop mode
utilized in the Haswell-EP architecture called cluster-of-die where the CPU and
shared L3 cache are effectively split in half, appearing as two non-uniform
memory access (NUMA) nodes to the operating system each with half the number of
cores and L3 cache. This splitting means that the latency to the L3 cache is
reduced, but the miss rate is also potentially higher due to the reduced size of
the cache. While there is a general trend of a decreasing L3 miss rate with
increasing threads, a few of the cases require further explanation. In the 6
process / 4 thread and 2 process / 12 thread cases, the miss rate is higher than
it would otherwise have been because processes may contain threads on both NUMA
nodes within a single socket and therefore cannot share data in the L3 cache. As
was seen for the Ivy Bridge node, having a single process causes half of the
memory accesses to cross the QPI bus which incurs a much higher
latency. Finally, we note that the large H-M model with tallies could not be
simulated with 24 MPI processes due to insufficient memory.
\begin{table}[htb]
  \centering
  \caption{Cache accesses and miss rates, bandwidth utilization, and observed
    performance for the small H-M benchmark on the Intel Xeon E5-2680 v3
    (Haswell) processor. All quantities followed by $\ddagger$ are on a per
    particle basis.}
  \label{tab:small-haswell}
  \footnotesize{
  \begin{tabular}{l*{8}{r}}
    \toprule
    & \multicolumn{2}{c}{L2 Instruction} & \multicolumn{2}{c}{L2 Data} &
    \multicolumn{2}{c}{L3} \\
    \cmidrule(r){2-3} \cmidrule(r){4-5} \cmidrule(r){6-7}
    \parbox{1.5cm}{Processes /\\Threads} & Accesses$^\ddagger$
    & \parbox[c]{1.2cm}{\centering Miss\\Rate (\%)} & Accesses$^\ddagger$
    & \parbox[c]{1.2cm}{\centering Miss\\Rate (\%)} & Accesses$^\ddagger$
    & \parbox[c]{1.2cm}{\centering Miss\\Rate (\%)}
    & \parbox[c]{1.3cm}{\centering Bandwidth$^\ddagger$\\(MiB/s)}
    & \parbox[c]{1.8cm}{\centering Tracking Rate\\(neutron/s)} \\
    \midrule
    \multicolumn{9}{c}{\textbf{No Tallies}} \\
    \midrule
    24 / 1 & 1314 & 40.1 & 26523 & 28.1 & 7957 & 19 & 9589 & 104189 \\
    12 / 2 & 1334 & 42.4 & 25649 & 33.9 & 9248 & 8.8 & 5436 & 109768 \\
    6 / 4 & 1369 & 42.4 & 25508 & 37 & 10010 & 9.2 & 4755 & 84991 \\
    4 / 6 & 1329 & 43.8 & 25472 & 38.4 & 10372 & 2.6 & 1797 & 109937 \\
    2 / 12 & 1366 & 42 & 25205 & 40.8 & 10868 & 11.1 & 5561 & 75256 \\
    1 / 24 & 1515 & 43.9 & 25888 & 40.6 & 11172 & 14.7 & 4177 & 41690 \\
    \midrule
    \multicolumn{9}{c}{\textbf{Tallies}} \\
    \midrule
    24 / 1 & 4460 & 29.1 & 60508 & 20 & 13392 & 23.7 & 9862 & 50866 \\
    12 / 2 & 4407 & 30.5 & 47419 & 27.5 & 14404 & 15.1 & 7406 & 55814 \\
    6 / 4 & 4457 & 29.7 & 47507 & 28.5 & 14873 & 14.9 & 6456 & 47695 \\
    4 / 6 & 4406 & 28.7 & 47115 & 28.7 & 14793 & 9 & 4809 & 59138 \\
    2 / 12 & 4451 & 30.8 & 47027 & 30.3 & 15596 & 15.4 & 6475 & 44061 \\
    1 / 24 & 4543 & 28.7 & 47757 & 29.2 & 15265 & 18.6 & 5534 & 31874 \\
    \bottomrule
  \end{tabular}
  }
\end{table}

\begin{table}[htb]
  \centering
  \caption{Cache accesses and miss rates, bandwidth utilization, and observed
    performance for the large H-M benchmark on the Intel Xeon E5-2680 v3
    (Haswell) processor. All quantities followed by $\ddagger$ are on a per
    particle basis.}
  \label{tab:large-haswell}
  \footnotesize{
  \begin{tabular}{l*{8}{r}}
    \toprule
    & \multicolumn{2}{c}{L2 Instruction} & \multicolumn{2}{c}{L2 Data} &
    \multicolumn{2}{c}{L3} \\
    \cmidrule(r){2-3} \cmidrule(r){4-5} \cmidrule(r){6-7}
    \parbox{1.5cm}{Processes /\\Threads} & Accesses$^\ddagger$
    & \parbox[c]{1.2cm}{\centering Miss\\Rate (\%)} & Accesses$^\ddagger$
    & \parbox[c]{1.2cm}{\centering Miss\\Rate (\%)} & Accesses$^\ddagger$
    & \parbox[c]{1.2cm}{\centering Miss\\Rate (\%)}
    & \parbox[c]{1.3cm}{\centering Bandwidth$^\ddagger$\\(MiB/s)}
    & \parbox[c]{1.8cm}{\centering Tracking Rate\\(neutron/s)} \\
    \midrule
    \multicolumn{9}{c}{\textbf{No Tallies}} \\
    \midrule
    24 / 1 & 2029 & 89 & 243020 & 82 & 201185 & 21.4 & 24388 & 9267 \\
    12 / 2 & 1986 & 88.8 & 202256 & 101.9 & 207830 & 16.1 & 19381 & 9470 \\
    6 / 4 & 1982 & 88.7 & 194227 & 106.4 & 208397 & 13.2 & 14026 & 8353 \\
    4 / 6 & 1950 & 88.6 & 192982 & 106.6 & 207414 & 7.5 & 10705 & 11244 \\
    2 / 12 & 2003 & 88.7 & 193545 & 106.6 & 208054 & 9.2 & 9729 & 8308 \\
    1 / 24 & 2135 & 88.7 & 193249 & 106.3 & 207256 & 9.9 & 7261 & 5794 \\
    \midrule
    \multicolumn{9}{c}{\textbf{Tallies}} \\
    \midrule
    24 / 1 & \multicolumn{8}{c}{---} \\
    12 / 2 & 6586 & 75.7 & 360578 & 87.4 & 320088 & 15.5 & 13274 & 4373 \\
    6 / 4 & 6278 & 75.6 & 326627 & 31.2 & 302693 & 13.5 & 11557 & 4648 \\
    4 / 6 & 6387 & 75.8 & 318472 & 91.2 & 295131 & 9.5 & 9408 & 5526 \\
    2 / 12 & 6286 & 75.4 & 308936 & 90.8 & 285141 & 10.9 & 9504 & 5024 \\
    1 / 24 & 6341 & 75.2 & 210156 & 91.3 & 287775 & 11.2 & 7872 & 3987 \\
    \bottomrule
  \end{tabular}
  }
\end{table}

%%%%%%%%%%%%%%%%%%%%%%%%%%%%%%%%%%%%%%%%%%%%%%%%%%%%%%%%%%%%%%%%%%%%%
\section{Conclusions}

In this study, the performance of a Monte Carlo neutron transport simulation of
a PWR benchmark was shown to be closely related to the number of accesses, miss
rates, and latencies to various levels of the memory hierarchy. Low-overhead
performance counters were collected by instrumenting the OpenMC Monte Carlo code
with calls to PAPI. By calculating an effective memory latency based on the
performance counter measurements, it was seen that the observed performance
scaled as the inverse of the effective memory latency.

The present study focused solely on MC simulations of neutron transport in
light-water reactor depletion problems, which are of great interest to the
reactor physics community. The characteristics of other problems may be
substantially different; in particular, other problems may not be as sensitive
to the effects of latency in the memory hierarchy. One could conjecture, e.g.,
that photon transport would be less sensitive because photon cross section data
us simpler in nature than neutron cross sections and consumes less
memory. Nevertheless, the conclusions of the present study may still
hold---rather than the primary drivers being the L2, L3, and main memory
performance, it might shift to the L1 and L2 caches only. Such conjectures would
need to be borne out through further studies.

It is important to recognize that there are many approximations and assumptions
inherent in our analysis. To being with, the PAPI performance counters we used
may not cover all events that lead to cache misses. For example,
\texttt{PAPI\_L3\_TCM} does not include L2 prefetches that miss in the L3
cache. Such events could be measured using more complex uncore\footnote{In some
  multi-core Intel architectures, the uncore subsystem refers to the components
  of a microprocessor that are not directly on the core but still essential to
  performance.} performance counters, but we were unable to use these counters
due to an insufficient Linux kernel version. We also observed apparent miss
rates greater than 100\% in \autoref{tab:large-haswell}, indicated that the set
of events covered by \texttt{PAPI\_L2\_DCM} is likely larger than the set of
events covered by \texttt{PAPI\_L2\_DCA}. In addition, we didn't account for
cache hits and misses in the TLB, which can sometimes have important effects on
performance. Lastly, in calculating an expected speedup, we assumed that the
performance completely governed by the memory latency; this is obviously
incorrect, as there are many other factors influencing performance. The fact
that, despite all these approximations and assumptions, the expected speedup
closely matched the actual speedup gives a strong indication about the relative
importance of memory latency on code performance, at least for the problems
tested.

An important outcome of the study is that the traditional logic that maximum
performance on a symmetric multiprocessor (SMP) node is always obtained by using
MPI processes only, if given enough memory, is by no means a hard-and-fast
rule. We saw in all cases that using more threads actually gives a performance
benefit due to the lower L3 miss rate. That being said, compiling a code without
threading enabled at all does allow the compiler to be more aggressive with
optimization, so some of performance gain may be offset by this ``threads
overhead''. It is hard to predict for a particular problem, computer platform,
and compiler what the optimal run configuration is, so application developers
and users are encouraged to test different combinations of processes and threads
to determine what is optimal for their particular problem.

Understanding the on-node scaling behavior for the current generation of
multi-core processors, featuring 2--20 heavyweight cores that rely heavily on
instruction-level parallelism, will help us to understand and reason about
expected performance on future architectures based on many-core processors,
featuring over 100 lightweight cores that rely primarily on thread-level and
data-level parallelism. It is anticipated that in order to attain exascale
supercomputing, many-core processors and/or GPUs are the only viable approach
due to their much-improved power efficiency.


%%%%%%%%%%%%%%%%%%%%%%%%%%%%%%%%%%%%%%%%%%%%%%%%%%%%%%%%%%%%%%%%%%%%%
\setlength{\baselineskip}{12pt}

\bibliographystyle{mc2015}
\bibliography{references}

\end{document}
